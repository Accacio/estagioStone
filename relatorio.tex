\documentclass[a4paper]{article}

\usepackage{enumerate}
%\usepackage{enumitem}
\usepackage{graphicx}
\graphicspath{{Figuras/}}
\usepackage{color}
\usepackage[cmex10]{amsmath}
\usepackage{array}
\usepackage{float}
\usepackage{multicol}
% \usepackage[latin1]{inputenc}
\usepackage[utf8]{inputenc} 
\usepackage[T1]{fontenc}
\usepackage[brazil]{babel}
%\usepackage[font=normalsize,format=plain,labelfont=bf,up,textfont=up,figurename=Figura,tablename=Tabela]{caption}
\usepackage[tablename=Tableau]{caption}
\usepackage{subcaption}
\usepackage[top=30mm, bottom=15mm, left=30mm, right=20mm]{geometry}
\usepackage{indentfirst}
\usepackage{fancyhdr}
% Font packages
\usepackage{amssymb}
\usepackage{amsfonts}
\usepackage{pgfgantt}
\usepackage{steinmetz}
\usepackage{rotating}
% Nice extra font package, e.g. \mathds{1}
\usepackage{dsfont}
\usepackage{color}
\usepackage{blindtext}
% Use multiple rows when writing tables
\usepackage{multirow}
\usepackage{booktabs}
\usepackage{bm}
\usepackage{bigstrut}
% Uncomment next line to make footnots per page
\usepackage{perpage}
% Uncoment next group of lines to create the table of contents for the PDF
\usepackage{hyperref}
\usepackage[toc,page]{appendix}
\usepackage{listings}
\usepackage{currfile}

\definecolor{darkblue}{rgb}{0,0,0.5}
\definecolor{darkblue}{rgb}{0,0,0.5}
\renewcommand{\title}{Titre}
\newcommand{\subtitle}{Subtitre}

\hypersetup{
    pdftitle={\title},
    pdfauthor={Author},
    bookmarksnumbered=true,     
    bookmarksopen=true,         
    bookmarksopenlevel=1,       
    colorlinks=true,
    linkcolor=black,
    filecolor=darkblue,  
    urlcolor=darkblue,  
    citecolor=darkblue,              
    pdfstartview=Fit,          
    pdfpagemode=UseOutlines,    % this is the option you were lookin for
    pdfpagelayout=TwoPageRight
}
\let\oldcontentsline\contentsline%
\renewcommand\contentsline[4]{%
    \oldcontentsline{#1}{\smash{\raisebox{1em}{\hypertarget{toc#4}{}}}#2}{#3}{#4}}

\newcommand\mysection[1]{\section[#1]{\protect\hyperlink{tocsection.\thesection}{#1}}\label{#1}}
\newcommand\mysubsection[1]{\subsection[#1]{\protect\hyperlink{tocsection.\thesection}{#1}}\label{#1}}
\newcommand\mysubsubsection[1]{\subsubsection[#1]{\protect\hyperlink{tocsection.\thesection}{#1}}\label{#1}}

\newcommand{\conteudo}{\tableofcontents\label{tocsection}}


\pagestyle{fancy}
\newif\ifdebug
\newcommand{\draft}{\debugtrue}
\newcommand{\final}{\debugfalse}
\newcommand\todo[1]{\ifdebug {\color{red}#1}\else \PackageError{}{FORGOT TO DO SOMETHING}{}\fi}
\newcommand\doing[1]{\ifdebug {\color{blue}#1}\fi}
\newcommand\warning[1]{\ifdebug {\color{red}#1}\fi}


\fancyhead[C]{\thepage}
\fancyhead[L]{}
\fancyhead[R]{}
%\fancyhead[RE]{\rightmark}
% \fancyhead[L]{\warning{DRAFT}}
% \fancyhead[R]{\warning{DEBUG ON}}

% \fancyfoot[L]{\warning{TURN DEBUG OFF}}
% \fancyfoot[R]{\warning{DRAFT}}
% \usepackage{showframe}
\fancyfoot[C]{}
\renewcommand{\headrulewidth}{0.pt}
\renewcommand{\footrulewidth}{0.pt}
\allowdisplaybreaks

\usepackage{chngcntr}
\counterwithin{figure}{section}


\newcommand{\figplaceholder}[1]{\ifdebug
	\begin{figure}[H]
		\begin{center}	
			\rule{8cm}{8cm}
			\caption{\color{red}placeholder}
			\label{fig:#1}
		\end{center}
	\end{figure}
\else
\PackageError{}{NO FIGURE}{}
\fi
}


\usepackage[acronym]{glossaries}\makeglossaries
\newcommand{\acr}[3]{\newacronym{#1}{#2}{#3}}
\newcommand{\symbl}[3]{\newglossaryentry{#1}{name = #2,	description = #3,}}

%%% Local Variables:
%%% mode: latex
%%% TeX-master: "rapport_stage"
%%% End:


%\symbl{teste2}{$\beta$}{beta}
%\acr{Supelec}{Supélec}{École Supérieure d'Électricité}

\acr{POS}{POS}{Point Of Service}
\acr{IETR}{IETR}{Institut d'Électronique et de Télécommunications de Rennes}
%
\acr{JPL}{JPL}{NASA Jet Propulsion Lab - Caltech - USA}
\acr{USP}{USP}{Universidade de São Paulo - Brasil}
\acr{Polimi}{POLIMI}{Politecnico Milano - Italia}

\acr{DPL}{DPL}{DIgSILENT Programming Language}
\acr{EDF}{EDF}{Électricité de France}
\acr{Enedis}{Enedis}{l'ancienne ERDF}


\acr{GUI}{GUI}{Interface Graphique d'utilisateur}

\acr{RMS}{RMS}{Root Mean Square - Moyenne quadratique}
\acr{EMT}{EMT}{Electro Magnetic Transient}
\symbl{Vnxx}{$V_{Nxx}$}{Tension du bus $ Nxx $}
\symbl{Qcxx}{$Q_{Cx-xx}$}{Puissance Réactive de la charge $ Cx-xx $}
\symbl{Qgdx}{$Q_{GDx}$}{Puissance Réactive du générateur $ GDx $}
\acr{CSV}{CSV}{Comma Separated Values}
\acr{DSL}{DSL}{DIgSILENT Simulation Language}

\makeindex
\final
\begin{document}
\thispagestyle{empty}
\large
\renewcommand{\figurename}{Figura}
\renewcommand{\tablename}{Tabela}
\pagenumbering{roman}

\begin{titlepage}
  \vspace{30mm}

\begin{center}
\textbf{Universidade Federal do Rio de Janeiro\\[4mm]
Escola Politéícnica\\[4mm]
Engenharia de Controle e Automacão\\[15mm]}
\textbf{Relatório de Atividades de Estágio Obrigatório}
\end{center}

\vfill

Autor:\rule[-0.5mm]{124mm}{0.2mm}\\
{\centering Rafael Accácio Nogueira\\[6mm]}

Orientador de Estágio:\rule[-0.5mm]{97mm}{0.2mm}\\

Orientador Acadêmico:\rule[-0.5mm]{96mm}{0.2mm}\\

Comissão de Estágio:\rule[-0.5mm]{100mm}{0.2mm}\\

Comissão de Estágio:\rule[-0.5mm]{100mm}{0.2mm}\\

Comissão de Estágio:\rule[-0.5mm]{100mm}{0.2mm}\\

\vfill


\begin{center}
\textbf{ECA\\[3mm]
Fevereiro/2019}
\end{center}
\end{titlepage}

%%% Local Variables:
%%% mode: latex
%%% TeX-master: "rapport_stage"
%%% End:


\newpage{ \pagestyle{empty} \qquad\newpage}
\conteudo
\newpage
\printglossary[type=\acronymtype,title=Lista de Acrônimos]
\phantomsection\addcontentsline{toc}{section}{Lista de Acrônimos}
\newpage

\pagenumbering{arabic}
\section{Resumo}

O objetivo deste documento é de brevemente resumir o trabalho
realizado pelo aluno Rafael Accácio Nogueira durante seu estágio
obrigatório na equipe de Machine Learning da empresa Stone Pagamentos,
uma adquirente de cartão de crédito, que realiza o aluguel
de máquinas de cartão de crédito, também conhecidas como \gls{POS}.
Seus clientes são dos mais variados, tanto em questão de escala quanto
de tipo de comércio. De um pequeno estabelecimento como uma padaria a
até uma cadeia de supermercados, ou postos de gasolina entre outros
tipos de estabelecimento A função desempenhada na empresa foi no
desenvolvimento de software tendo feitos contribuições para alguns
projetos da equipe, aumentando o número de funcionalidades do projeto,
como controle de acesso, por exemplo. Dentro do objetivo principal
está descrever as tarefas desempenhas e as detalhando sempre que
possível, explicando os projetos, o que com eles foi aprendido (as
tecnologias e as metodologias), o processo de aprendizagem e como esse
aprendizado contribuiu para a formação de um engenheiro mais
multidisciplinar.  Além de, é claro, mostrar as dificuldades
encontradas durante o período, e que foi feito para serem
ultrapassadas e como.

~\\
Palavras-chave:
\begin{itemize}
\item Adquirente
\item Big Data
\item Análise de Dados
\item Banco de Dados
\item Controle de Acesso
\end{itemize}

\pagebreak
\section{Introdução}
Este relatório é dividido em 4 partes : Fundamentos Teóricos e
Práticos, Resultados, Discussão, Conclusão.

Primeiro os fundamentos teóricos e práticos para o
desenvolvimento dos projetos são apresentados. Depois os resultados
obtidos são indicados. Na sessão seguinte é feita uma pequena
discussão sobre esses resultados e na conclusão o que foi levado desse
estágio para a formação como engenheiro.

\section{Fundamentos Teóricos e Práticos}
A maior parte dos fundamentos teóricos requeridos para a execução da função
desempenhada no estágio, foram adquiridos basicamente a partir das
matérias cursadas na faculdade, Algoritmos de Programação e Linguagens
de Programação, cursadas nos dois primeiros períodos da
faculdade, onde o básico de programação foi aprendido em diversas
linguagens possibilitando o fácil aprendizado de outras sintaxes.

Outros fundamentos teóricos tiveram que ser aprendidos na própria
empresa, como por exemplo alguns fundamentos sobre mercado financeiro
e o funcionamento do ciclo de transações em um pagamento usando cartão
de crédito, que possui como atores o portador do cartão de crédito (cliente do
estabelecimento), o estabelecimento, a adquirente, a bandeira do
cartão de crédito e o Emissor (geralmente um banco).

Nas primeiras semanas, os fundamentos teóricos do funcionamento da
equipe e das linguagens, softwares e ferramentas utilizadas para a realização dos projetos foram
ensinados pelo orientador do estágio e outros membros da equipe, além
de um curso de Scala disponibilizado para o aprendizado da sintaxe da
linguagem.

Programas, Linguagens e Ferramentas utilizados durante o estágio:

\begin{itemize}
\item IntelliJ IDEA
\item SQL Management Studio
\item Postman
\item RabbitMQ
\item Grafana
\item Kafka
\end{itemize}

Alguns outros programas foram utilizados como alternativas para alguns
destes supracitados para Linux a fim de aumentar o aprendizado no tema, alguns deles
são:
\begin{itemize}
\item Emacs
\item DBeaver 
\item curl
\item Insomnia REST Client
\end{itemize}
\pagebreak
\section{Resultados} 
Alguns dos resultados foram pequenas melhorias na aplicação Zephyros CEP, que
é utilizada como infraestrutura para o trânsito de informações de
transações entre outras utilizadas por basicamente todas as outras
aplicações da equipe.

Outro resultado foi a implementação de regras de filtragem de Clientes da empresa
presente na aplicação Ogma-Rules, para uso posterior por outras
equipes a fim de verificar clientes em situação de churn ou pré-churn,
a fim de tê-los como alvo, para combater a saída destes mesmos
clientes da empresa. Churn é
a nomenclatura dada para um cliente que deixa de utilizar por
quaisquer motivos os \gls{POS} da empresa enquanto o contrato ainda está
vigente, esse hiato entre utilizações pode ser um indicativo de uma
eventual saída, que visa ser combatida, através da conversa com o cliente para assim
melhor entendê-lo e melhor atender suas expectativas quanto a empresa.

Além disso outro resultado é o controle de acesso de usuários
para a aplicação Odin, que é usado para visualização de transações em
tempo real usando Grafana para fazer um dashboard para plotar gráficos
e cartões indicativos dos dados. Esse controle de acesso foi criado
utilizando alto grau de generalismo de modo que no futuro pode ser
utilizado até mesmo por outras aplicações, tanto as correntes quanto
as futuras. 

Um outro resultado embora pequeno foi o desenvolvimento de novas
funções da biblioteca gim-scala-client, que é um wrapper do GIM em
scala, possibilitando a gerência de usuários e recursos das
aplicações a partir do GIM, ferramenta da empresa para autenticação de
usuários e aplicações.

Todos os projetos foram feitos em Scala, usando a ferramenta Git para gestão de
versão dos projetos, sendo utilizado um repositório privado no Github
da empresa.

\section{Discussão}

Embora alguns dos conhecimentos utilizados para as tarefas bases foram
aprendidos em poucas matérias, as vezes visto somente em algumas
eletivas, alguns conhecimentos aprendidos em matérias como cálculos,
álgebra linear, lógica matemática, físicas, otimização e até mesmo
Desenho de Engenharia foram uteis para resolução de alguns
problemas. Logo, esse período do estágio ajudou a mostrar uma outra
faceta de onde e como um Engenheiro de Controle e Automação é capaz de
trabalhar, além de uma experiência em uma empresa de uma área nem
sempre muito comentada durante o curso (mercado financeiro).

Algumas dificuldades foram encontradas durante o processo, uma vez que
certos temas não haviam sido explicitados durante as aulas, por serem
muito específicos tanto para a área do mercado financeiro quanto para
o desenvolvimento de aplicações. Essas dificuldades foram vencidas ao
utilizar fontes de pesquisa como a internet, livros que foram
disponibilizados pelos companheiros de equipe e é claro sempre que
possível através das interações com os outros membros da equipe mais
experientes.

\section{Conclusão}

O período serviu para aumentar o aprendizado de métodos e técnicas
utilizadas no mercado, assim como ferramentas utilizadas para Big Data
e Data Science, que podem ser utilizados também para controle de Smart
Grids entre outras coisas, tão pedidos no mercado atual onde
basicamente todos os fluxos de dados são massivos, precisando assim de
ferramentas capazes de suportá-los como Spark, Flink e etc para
processamento desses dados que geralmente vêm em batelada e o fluxo de
dados distribuídos provenientes de muitos pontos diferentes espalhados
pelo planeta.

\end{document}
