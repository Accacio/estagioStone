\usepackage{enumerate}
%\usepackage{enumitem}
\usepackage{graphicx}
\graphicspath{{Figuras/}}
\usepackage{color}
\usepackage[cmex10]{amsmath}
\usepackage{array}
\usepackage{float}
\usepackage{multicol}
% \usepackage[latin1]{inputenc}
\usepackage[utf8]{inputenc} 
\usepackage[T1]{fontenc}
\usepackage[brazil]{babel}
%\usepackage[font=normalsize,format=plain,labelfont=bf,up,textfont=up,figurename=Figura,tablename=Tabela]{caption}
\usepackage[tablename=Tableau]{caption}
\usepackage{subcaption}
\usepackage[top=30mm, bottom=15mm, left=30mm, right=20mm]{geometry}
\usepackage{indentfirst}
\usepackage{fancyhdr}
% Font packages
\usepackage{amssymb}
\usepackage{amsfonts}
\usepackage{pgfgantt}
\usepackage{steinmetz}
\usepackage{rotating}
% Nice extra font package, e.g. \mathds{1}
\usepackage{dsfont}
\usepackage{color}
\usepackage{blindtext}
% Use multiple rows when writing tables
\usepackage{multirow}
\usepackage{booktabs}
\usepackage{bm}
\usepackage{bigstrut}
% Uncomment next line to make footnots per page
\usepackage{perpage}
% Uncoment next group of lines to create the table of contents for the PDF
\usepackage{hyperref}
\usepackage[toc,page]{appendix}
\usepackage{listings}
\usepackage{currfile}

\definecolor{darkblue}{rgb}{0,0,0.5}
\definecolor{darkblue}{rgb}{0,0,0.5}
\renewcommand{\title}{Titre}
\newcommand{\subtitle}{Subtitre}

\hypersetup{
    pdftitle={\title},
    pdfauthor={Author},
    bookmarksnumbered=true,     
    bookmarksopen=true,         
    bookmarksopenlevel=1,       
    colorlinks=true,
    linkcolor=black,
    filecolor=darkblue,  
    urlcolor=darkblue,  
    citecolor=darkblue,              
    pdfstartview=Fit,          
    pdfpagemode=UseOutlines,    % this is the option you were lookin for
    pdfpagelayout=TwoPageRight
}
\let\oldcontentsline\contentsline%
\renewcommand\contentsline[4]{%
    \oldcontentsline{#1}{\smash{\raisebox{1em}{\hypertarget{toc#4}{}}}#2}{#3}{#4}}

\newcommand\mysection[1]{\section[#1]{\protect\hyperlink{tocsection.\thesection}{#1}}\label{#1}}
\newcommand\mysubsection[1]{\subsection[#1]{\protect\hyperlink{tocsection.\thesection}{#1}}\label{#1}}
\newcommand\mysubsubsection[1]{\subsubsection[#1]{\protect\hyperlink{tocsection.\thesection}{#1}}\label{#1}}

\newcommand{\conteudo}{\tableofcontents\label{tocsection}}


\pagestyle{fancy}
\newif\ifdebug
\newcommand{\draft}{\debugtrue}
\newcommand{\final}{\debugfalse}
\newcommand\todo[1]{\ifdebug {\color{red}#1}\else \PackageError{}{FORGOT TO DO SOMETHING}{}\fi}
\newcommand\doing[1]{\ifdebug {\color{blue}#1}\fi}
\newcommand\warning[1]{\ifdebug {\color{red}#1}\fi}


\fancyhead[C]{\thepage}
\fancyhead[L]{}
\fancyhead[R]{}
%\fancyhead[RE]{\rightmark}
% \fancyhead[L]{\warning{DRAFT}}
% \fancyhead[R]{\warning{DEBUG ON}}

% \fancyfoot[L]{\warning{TURN DEBUG OFF}}
% \fancyfoot[R]{\warning{DRAFT}}
% \usepackage{showframe}
\fancyfoot[C]{}
\renewcommand{\headrulewidth}{0.pt}
\renewcommand{\footrulewidth}{0.pt}
\allowdisplaybreaks

\usepackage{chngcntr}
\counterwithin{figure}{section}


\newcommand{\figplaceholder}[1]{\ifdebug
	\begin{figure}[H]
		\begin{center}	
			\rule{8cm}{8cm}
			\caption{\color{red}placeholder}
			\label{fig:#1}
		\end{center}
	\end{figure}
\else
\PackageError{}{NO FIGURE}{}
\fi
}


\usepackage[acronym]{glossaries}\makeglossaries
\newcommand{\acr}[3]{\newacronym{#1}{#2}{#3}}
\newcommand{\symbl}[3]{\newglossaryentry{#1}{name = #2,	description = #3,}}

%%% Local Variables:
%%% mode: latex
%%% TeX-master: "rapport_stage"
%%% End:
